\documentclass[a4paper,12pt,titlepage,finall]{article}

\usepackage[T1,T2A]{fontenc}     % форматы шрифтов
\usepackage[utf8x]{inputenc}     % кодировка символов, используемая в данном файле
\usepackage[russian]{babel}      % пакет русификации
\usepackage{tikz}                % для создания иллюстраций
\usepackage{pgfplots}            % для вывода графиков функций
\usepackage{geometry}		 % для настройки размера полей
\usepackage{indentfirst}         % для отступа в первом абзаце секции
\usepackage{multirow}            % для таблицы с результатами

% выбираем размер листа А4, все поля ставим по 3см
\geometry{a4paper,left=30mm,top=30mm,bottom=30mm,right=30mm}

\setcounter{secnumdepth}{0}      % отключаем нумерацию секций

\usepgfplotslibrary{fillbetween} % для изображения областей на графиках

\begin{document}
% Титульный лист
\begin{titlepage}
    \begin{center}
	{\small \sc Московский государственный университет \\имени М.~В.~Ломоносова\\
	Факультет вычислительной математики и кибернетики\\}
	\vfill
	{\Large \sc Отчет по заданию №1}\\
	~\\
	{\large \bf <<Методы сортировки>>}\\ 
	~\\
	{\large \bf Вариант 2 / 2 / 1 / 5}
    \end{center}
    \begin{flushright}
	\vfill {Выполнил:\\
	студент 102  группы\\
	Никитин~В.~В.\\
	~\\
	Преподаватель:\\
	Смирнов~А.~В.}
    \end{flushright}
    \begin{center}
	\vfill
	{\small Москва\\2017}
    \end{center}
\end{titlepage}

% Автоматически генерируем оглавление на отдельной странице
\tableofcontents
\newpage

\section{Постановка задачи}

Необходимо реализовать два метода сортировки массива чисел и провести их экспериментальное сравнение. Для каждого из
реализуемых методов необходимо предусмотреть возможность работы с массивами длины от 1 до N (N ≥ 1).
При реализации каждого метода вычислить число сравнений элементов и число перемещений (обменов) элементов.\\
~\\
Сравнение методов сортировки необходимо проводить на одних и тех же исходных массивах, при этом
следует рассмотреть массивы разной длины. Для вариантов с фиксированным значением N рассмотреть, как
минимум, n = 10, 20, 50, 100. Для вариантов с динамическим выделением памяти — n = 10, 100, 1000, 10000.
Генерация исходных массивов для сортировки реализуется отдельной функцией, создающей в зависимости от
заданного параметра и заданной длины конкретный массив, в котором:
\begin{itemize}
\item элементы уже упорядочены (1);
\item элементы упорядочены в обратном порядке (2);
\item расстановка элементов случайна (3, 4).
\end{itemize}
Результаты экспериментов оформить на основе нескольких запусков программы в виде таблицы.\\
~\\
Тип данных: 64-битные целые числа (long long int)\\
Вид сортировки: числа упорядочиваются по невозрастанию\\
Методы сортировки: метод <<пузырька>>, пирамидальная сортировка.\\
~\\
Более подробное описание методов сортировки можно прочесть в программе, либо в разделе <<Структура программы и спецификация функций>>\\

\newpage

\section{Результаты экспериментов}

Элементы массива с номером 1 уже упорядочены (лучший случай), с номером 2 расположены в обратном порядке (худший случай), с номерами 3-4 расположены в случайном порядке.

\begin{table}[h]
\centering
\begin{tabular}{|c|c|c|c|c|c|c|c|}
    \hline
    \multirow{2}{*}{\textbf{n}} & \multirow{2}{*}{\textbf{Параметр}} & \multicolumn{4}{|c|}{\textbf{Номер сгенерированного массива}} & \textbf{Среднее} \\
    \cline{3-6}
    & & \parbox{1.5cm}{\centering 1} & \parbox{1.5cm}{\centering 2} & \parbox{1.5cm}{\centering 3} & \parbox{1.5cm}{\centering 4} & \textbf{значение} \\
    \hline
    \multirow{2}{*}{10} & Сравнения &45 &45 &45 &45 &45 \\
    \cline{2-7}
                        & Перемещения &0 &45 &13 &29 &22 \\
    \hline
    \multirow{2}{*}{100} & Сравнения &4990 &4990 &4990 &4990 &4990 \\
    \cline{2-7}
                         & Перемещения &0 &4990 &2676 &2551 &2555 \\
    \hline
    \multirow{2}{*}{1000} & Сравнения &499500 &499500 &499500 &499500 &499500 \\
    \cline{2-7}
                          & Перемещения &0 &499500 &265677 &245980 &254038 \\
    \hline
    \multirow{2}{*}{10000} & Сравнения &49995000 &49995000 &49995000 &49995000 &49995000 \\
    \cline{2-7}
                           & Перемещения &0 &49995000 &25076575 &25485962 &25265508 \\
    \hline
\end{tabular}
\caption{Результаты работы метода <<пузырька>>}
\end{table}
\begin{table}[h]
\centering
\begin{tabular}{|c|c|c|c|c|c|c|c|}
    \hline
    \multirow{2}{*}{\textbf{n}} & \multirow{2}{*}{\textbf{Параметр}} & \multicolumn{4}{|c|}{\textbf{Номер сгенерированного массива}} & \textbf{Среднее} \\
    \cline{3-6}
    & & \parbox{1.5cm}{\centering 1} & \parbox{1.5cm}{\centering 2} & \parbox{1.5cm}{\centering 3} & \parbox{1.5cm}{\centering 4} & \textbf{значение} \\
    \hline
    \multirow{2}{*}{10} & Сравнения &60 &52 &58 &62 &58 \\
    \cline{2-7}
                        & Перемещения &38 &29 &33 &34 &34 \\
    \hline
    \multirow{2}{*}{100} & Сравнения &1288 &1134 &1232 &1228 &1221 \\
    \cline{2-7}
                         & Перемещения &738 &598 &676 &677 &672 \\
    \hline
    \multirow{2}{*}{1000} & Сравнения &19660 &17968 &18786 &18820 &18809 \\
    \cline{2-7}
                          & Перемещения &10032 &10710 &9312 &10030 &10077 \\
    \hline
    \multirow{2}{*}{10000} & Сравнения &264526 &246744 &255330 &255386 &255497 \\
    \cline{2-7}
                           & Перемещения &141424 &126976 &134258 &134158 &134204 \\
    \hline
\end{tabular}
\caption{Результаты работы пирамидальной сортировки}
\end{table}
~\\
Стоить заметить, что метод <<пузырька>> сравнивает значения элементов массива \((n-1)*n/2\) раз, количество перемещений элементов в лучшем случае равно 0, в худшем равно \((n-1)*n/2\). Сложность алгоритма O(\(n^2\)).\\
~\\
Совершенно иные данные выдает нам метод пирамидальной сортировки. Функция RightPyramid просеивает элемент через дерево, сложность O(\(log(n)\)), повторяется данное действие O(\(n\)) раз. Сложность алгоритма O(\(n~log(n)\)).\\
~\\
Исходя из тестов можно сказать, что пирамидальная сортировка однозначно лидирует на больших наборах данных. Метод "пузырька" имеет преимущество только на полностью отсортированном наборе, что на практике встречается довольно редко.\\

\newpage

\section{Структура программы и спецификация функций}

~\\
int main (void);\\
Специальная функция. Является начальной точкой выполнения программы.\\
~\\
void CreateArray(int param, int n);\\
Функция, которая создает статический массив с элементами, порядок которых зависит от параметра (подробнее о параметре на стр. 2).\\
Входные данные: 'n'\ - количество элеметов в массиве, 'param'\ - параметр.\\
~\\
void BubbleSort(long long int a[], int n);\\
Функция, которая сортирует массив методом "пузырька". Упрощенно: сравнивает элементы a[i], a[i+1] и, если a[i]<a[i+1], то меняет их местами.\\
Входные данные: 'a'\ - массив, 'n'\ - количество элементов массива\\
~\\
void PyramidSort(long long int a[], int n);\\
Функция, которая сортирует массив с помощью метода пирамидальной сортировки. Создает правильную пирамиду, удаляет высоту, переносит на место высоты последний элемент, заново создает правильную пирамиду.\\
Входные данные: 'a'\ - массив, 'n'\ - количество элементов массива\\
~\\
void RightPyramid(long long int a[], int k, int m);\\
Функция, которая создает правильную пирамиду. Просеиваем a[k] элемент через пирамиду.\\
Входные данные: 'a'\ - массив, 'k'\ - последний элемент пирамиды, 'n'\ - первый элемент пирамиды\\

\newpage

\section{Отладка программы, тестирование функций}

В процессе отладки были устранены недочеты (пропущенные знаки, табуляция) и неиспользуемые переменные, добавлены комментарии к коду.\\
Тестирование программы проводилось на 4 массивах: \{0\}, \{10000, 9999, 9998, 9997\}, \{0,1,2,4\}, \{-2126464088, 737385984, -1010079636\} \\

\newpage

\section{Анализ допущенных ошибок}

Ошибок не было допущено.\\

\newpage
\begin{raggedright}
\addcontentsline{toc}{section}{Список цитируемой литературы}
\begin{thebibliography}{99}
\bibitem{cs} Кормен Т., Лейзерсон Ч., Ривест Р, Штайн К. Алгоритмы: построение и анализ.
    Второе издание.~--- М.:<<Вильямс>>, 2005.
\bibitem{cs} Белеванцев А.А. Конспект лекций по программированию для студентов 1 курса ВМК МГУ, 2016-2017.
\end{thebibliography}
\end{raggedright}
\end{document}
